\appendix

\section{Calculations}
\label{appendix:calculations}

\subsection{Moment of inertia of rectangle}
\begin{equation}
	\label{eq:rect_moment_long}
	\begin{split}
		I_{\text{rect}} & =
		\rho \int_{-\frac{h}{2}}^{\frac{h}{2}} \int_{-\frac{w}{2}}^{\frac{w}{2}}  x^2 + y^2 \diff x \diff y           \\
		& = 4 \rho \int_{0}^{\frac{h}{2}} \int_{0}^{\frac{w}{2}}  x^2 + y^2 \diff x \diff y           \\
		& = 4 \rho \int_{0}^{\frac{h}{2}} \Biggl[\frac{1}{3} x^3 + x y^2 \Biggr]_{0}^{\frac{w}{2}} dy \\
		& = 4 \rho \int_{0}^{\frac{h}{2}} \frac{1}{3} \frac{w^3}{8} + \frac{w}{2} y^2 dy              \\
		& = 2 \rho \int_{0}^{\frac{h}{2}} \frac{w^3}{12} + w y^2 dy                                   \\
		& = 2 \rho \Biggl[ \frac{w^3}{12}  y + \frac{w}{3} y^3 \Biggr]_{0}^{\frac{h}{2}}              \\
		& = 2 \rho \frac{w}{3} \Biggl[ \frac{w^2}{4}y + y^3 \Biggr]_{0}^{\frac{h}{2}}                 \\
		& = 2 \rho \frac{w}{3} \left( \frac{w^2}{4}\frac{h}{2} + \frac{h^3}{8} \right)                \\
		& = \rho \frac{w}{3} \left( \frac{w^2}{4}h + \frac{h^3}{4} \right)                            \\
		& = \frac{\rho wh}{12} \left( w^2 + h^3 \right)                                               \\
	\end{split}
\end{equation}

\newpage
\subsection{Moment of inertia of sub-triangle of regular polygon}
Before starting the calculations, it is to be noted that according to Figure
\ref{fig:subtriangle}, we have that
$$ \tan\left(\frac{\theta}{2}\right) = \frac{\frac{l}{2}}{h} = \frac{l}{2h} $$
it will be useful to simplify the result of the integral.

\begin{equation}
	\label{eq:subtriangle_moment_long}
	\begin{split}
		I_{T} &= \rho \int_0^h\int_{-\frac{lx}{2h}}^{\frac{lx}{2h}}x^2 + y^2 \diff y\diff x\\
		&= 2\rho \int_0^h\int_0^{\frac{lx}{2h}} x^2 + y^2 \diff y\diff x\\
		&= 2\rho \int_0^h \Biggl[x^2y + \frac{1}{3} y^3\Biggr]_0^{\frac{lx}{2h}} \diff x\\
		&= 2\rho \int_0^h x^2 \frac{lx}{2h} + \frac{1}{3} \frac{l^3x^3}{8h^3} \diff x\\
		&= 2\rho \left(\frac{l}{2h} + \frac{l^3}{24h^3}\right) \int_0^h x^3 \diff x\\
		&= 2\rho \left(\frac{l}{2h} + \frac{l^3}{24h^3}\right)  \Biggl[ \frac{1}{4} x^4\Biggr]_0^h \\
		&= \frac{h^4\rho}{2} \left(\frac{l}{2h} + \frac{l^3}{24h^3}\right) \\
		&= \frac{\rho lh^3}{4} \left(1 + \frac{l^2}{12h^2}\right) \\
		&= \frac{m_T h^2}{2} \left(1 + \frac{l^2}{12h^2}\right) \\
		&= \frac{m_T}{2} \frac{l^2}{4 \tan^2\left(\frac{\theta}{2}\right)}\left(1 + \frac{4 \tan^2\left(\frac{\theta}{2}\right)}{12}\right) \\
		&= \frac{m_Tl^2}{24} \left(1 + 3\cot^2\left(\frac{\theta}{2}\right)\right)
	\end{split}
\end{equation}

\newpage
\subsection{Moment of inertia of sub-triangle of arbitrary polygon} Recall
equation \ref{eq:r} defines
$$ \vec r = \alpha \vv{CA} + \beta \alpha \vv{AB} $$
\begin{equation}
	\label{eq:subtriangle_arbitrary_moment_long}
	\begin{split}
		I_{T_i} &= \rho \int_0^1 \int_0^1 \vec r^2 hb \alpha  \diff \alpha \diff \beta\\
		&=\rho hb\int_0^1 \int_0^1 \left(\alpha \vv{CA} + \beta \alpha \vv{AB}\right)^2 \alpha  \diff \alpha \diff \beta\\
		&=\rho hb\int_0^1 \int_0^1 \left(\alpha^2 \vv{CA}^2 + 2 \alpha^2 \beta \vv{AB} \cdot \vv{CA} + \alpha^2 \beta^2 \vv{AB}^2\right) \alpha  \diff \alpha \diff \beta\\
		&=\rho hb\int_0^1 \int_0^1 \alpha^3 \left(\vv{CA}^2 + 2 \beta \vv{AB} \cdot \vv{CA} + \beta^2 \vv{AB}^2\right) \diff \alpha \diff \beta\\
		&=\rho hb\int_0^1 \Biggl[\frac{1}{4} \alpha^4 \left(\vv{CA}^2 + 2 \beta \vv{AB} \cdot \vv{CA} + \beta^2 \vv{AB}^2\right) \Biggr]_0^1\diff \beta\\
		&= \frac{\rho hb}{4}\int_0^1 \beta^2 \vv{AB}^2 + 2 \beta \vv{AB} \cdot \vv{CA} + \vv{CA}^2 \diff \beta\\
		&= \frac{\rho hb}{4} \Biggl[\frac{1}{3} \beta^3 \vv{AB}^2 + \beta^2 \vv{AB} \cdot \vv{CA} + \beta\vv{CA}^2 \Biggr]_0^1\\
		&= \frac{\rho hb}{4} \left(\frac{1}{3}\vv{AB}^2 + \vv{AB} \cdot \vv{CA} + \vv{CA}^2\right) \\
	\end{split}
\end{equation}

\newpage
\subsection{Solving for impulse parameter}
\label{app:impulse_long}
We start with equation \ref{eq:vp2n}:
\begin{equation*}
	\begin{split}
		\vec v_{p2} \cdot \vec n &= - e \vec v_{p1} \cdot \vec n\\
		\left(\vec v_{ap2} - \vec v_{bp2}\right)\cdot \vec n &= - e \vec v_{p1} \cdot \vec n\\
		\left( \vec v_{a2} + \omega_{a2} \times \vec r_{ap} - \vec v_{b2} - \omega_{b2} \times \vec r_{bp}
		\right)\cdot \vec n &= - e \vec v_{p1} \cdot \vec n\\
	\end{split}
\end{equation*}

We now expand the bracket on the left-hand side using the equations \ref{eq:va2}
- \ref{eq:omega_b2} and then simplify with equation \ref{eq:vp1}.

\[
	\begin{split}
		\left( \vec v_{a1} +  \frac{j\vec n}{m_a} + \omega_{a1} + \frac{\vec r_{ap} \times j\vec n}{I_a} \times \vec r_{ap} - \vec v_{b1} +  \frac{j\vec n}{m_b} - \omega_{b1} + \frac{\vec r_{bp} \times j\vec n}{I_b} \times \vec r_{bp} \right)\cdot \vec n &= - e \vec v_{p1} \cdot \vec n\\
		\left( \frac{j\vec n}{m_a} + \frac{\vec r_{ap} \times j\vec n}{I_a} \times \vec r_{ap} +  \frac{j\vec n}{m_b} + \frac{\vec r_{bp} \times j\vec n}{I_b} \times \vec r_{bp} \right)\cdot \vec n &= - (1+e) \vec v_{p1} \cdot \vec n\\
	\end{split}
\]
Using the triple scalar product rule, we can derive that
\[
	j \left( 1/m_a + 1/m_b + \frac{\left(( \vec r_{ap} \times \vec n
			)^2\right)}{I_a} + \frac{\left( \vec r_{bp} \times \vec n \right)^2}{I_b}
	\right) = -(1+e) \vec v_{p1} \cdot \vec n
\]
and therefore
\begin{equation}
	j = \frac{ - (1+e) \cdot \vec v_{ap1} \cdot \vec n }{\frac{1}{m_a} + \frac{1}{m_b} +
		\frac{\left( \vec r_{ap} \times \vec n \right)^2}{I_a} + \frac{\left( \vec
			r_{bp} \times \vec n \right)^2}{I_b}}
\end{equation}
